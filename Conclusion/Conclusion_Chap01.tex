\section{Chapter \ref{chap:BLE}}
Chapter \ref{chap:BLE} took more space in terms of the RF environment design for the BMS. The Chapter includes Bluetooth module design, Antenna design, RF layout design, etc. In this chapter, I have covered how to pick Bluetooth solutions for BMS applications and what factors we need to keep in mind to select a Bluetooth solution. So far so forth picking a BLE solution was successful, and I have ended with BLUeNRG355 and Nordic Bluetooth stacks. What I have not strongly assured, is why only BLUeNRG355 and Nordic. Well it's not obligatory to go to only these two solutions we have plenty of BLE solutions in the market and the reason behind selecting these two solutions are they are well-established in the Bluetooth world and their documentation is very much self-explanatory.
Users can select any BLE stack depending on their application requirements I have opted for these two solutions based on the project requirements.

Later sections in chapter 1 talk about antenna selection and antenna designing, what makes the reader disconnected from this section? Is mathematical design of the antennas.. Yes I have not included the mathematical calculations because in the proper solutions it is hard to pick new experimental RF antenna shapes the intent of this project purely was to make the BLE communication-based BMS and the main motive was to bring up the nice RF layout using the proprietary solutions. Keeping these requirements in mind I have picked standard MIFA and PIFA antennas and I took an opportunity to guide a reader to literature for more design theory. Yet, I have not missed a chance to discuss the results and compare different antenna results for picking the right antenna for BMS applications.

Further, sections in Chapter 1 are explored in much more detail the RF layout design and precautions we need to take to prevent the RF layout affecting from parasitic components.
After looking at Bluetooth much deeper we should relate this solution to the BMS, which is the most beautiful combination. I have tried to integrate the BLE module with the existing setup, which came out as a very successful solution for the BMS.

\subsection{Future of BLE in BMS}
So far and so forth BLE proved it is the most reliable solution to the BMS application, but I don't suggest users rely only on the BLE solution. We can be much more explorers with different disciplines like Wi-Fi, Zigbee, and even other proprietary protocols.
I have not stressed the security of BLE, the data management, and square in the BLE these are the topics that could open new doors in the BLE world for the BMS. There is also an open opportunity to explore the BLE integrating with the onboard solution of the BMS (now it is a module-based approach). Doing this could challenge the existing BLE layout and BMS ckt's(due to the parasitics, noise, and power supply issues even many other issues). Nevertheless, the proposed BLE stack and design strategies sustained strong and consistent for our BMS application.

I have a very interesting research idea to test this BLE solution with a large setup and the real battery module setup, like in EV battery packs and I could see the new situation (interference, reflections, etc...) and how reliable this solution could be. Since this is not the scope of my project, I will this enthusiastic idea for the future of my research.