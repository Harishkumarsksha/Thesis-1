\section*{Conclusion of Chapter \ref{ch:Battery_Modeling_Emulation}}

Batteries are one of the greatest inventions for mankind, and they have two kinds of views they can be greatest to decrease global warming, and they can be more dangerous if you do not handle them properly. Since batteries are more hazardous in terms of chemicals, and they are very sensitive to the voltages it is essential to give little extra care. It is more often engineers who work on BMS need to do a lot of testing on the batteries and there is a high chance that engineers can encounter some unpleasant events. So, the BMS engineers come up with something beautiful, more effective, and less dangerous. Chapter \ref{ch:Battery_Modeling_Emulation} attempted to address this issue. The proposal was to model the battery as a real battery and emulate it with the lab setup.
\\\\
I have partitioned chapter 4 into two parts first is battery modeling, and the second is implementing the lab setup. The battery modeling section is very much descriptive it explains from scratch how to model the battery and brings up the model parameters from the soc characteristics graph. It is gorgeous, I love the way the battery is modeled with the known circuit components. Not an exaggeration I went through the simple resistive model to the two-time constant model, the future of this modeling could go even high for more accurate battery modeling.\\\\
\\
Part II of chapter 4 is completely based on part I, implemented in the lab setup. I have proposed a power analyzer architecture for the most robust battery modeling and testing in the lab setup. The battery model implementation took place entirely with the python script. The results of this implementation were phenomenally interesting, I took quite a long word to elaborate on the results, please have look.\\
The power analyzer implementation is taken place through the script, but it is not obligatory to go only through this script it is the user's choice. They can make an independent decision, I set up this script based on the instruments that are available to me. The users can make scripts depending on the instruments in their lab, but the power analyzer architecture holds good universally for all BMS lab testing.\\

I have only described major configurations for instruments since scripts take loads of coding lines I have uploaded all the scripts to GitHub and addressed respective links in the miscellaneous.

