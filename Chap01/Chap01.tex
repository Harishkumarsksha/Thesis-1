%Formatting Guidelines for Writing Dissertation.
\chapter*{Introduction}\label{intro}

After more than a century of development, thousands of homes have adopted cars, which have evolved from luxury goods to necessities in modern society since the world's first car was launched. The three pillars of energy efficiency, environmental preservation, and safety are the constant themes in the evolution of automotive technology. Pure electric vehicles, or new energy vehicles, have steadily gained industry attention due to their energy-saving and environmental benefits. Since, the turn of the twenty-first century, when oil prices began to rise and environmental pollution issues, such as "haze," became worse. %\cite{Real_Time_SOC_Estimation_Based_On_EKF_And_UKF_PTorin_Lei}.

At present, the world heading towards safe and green energy. One of the revolutionary research in energy saving and decreasing global warming is EVs. As the core component of energy-saving vehicles (EVs), the development of battery technology is the key to the industrialization of new energy vehicles. 

The promotion of pure electric vehicles has been hampered by people's worries about the range of electric cars, even though all major automakers are working to promote new energy models. Additionally, the frequent incidents of spontaneous combustion of electric cars in recent years have caused people to pay extra attention to the safety of electric cars. The safe and effective operation of the power battery in electric vehicles relies on a precise assessment of the battery status.

Among the EVs world, Lithium-ion batteries (LIB)  are uncrowned batteries, because of their energy density, reliability, and size constraints.  It is common knowledge that for safe use and enduring performance, a battery management system (BMS) is essential. However, what is less widely known is that during each discharge cycle, some unused energy remains in the battery. This is not due to technical design considerations, but to a phenomenon called cell-to-cell variation (C2CV). Due to physical characteristics, individual battery cells in a stack of multiple cells vary in capacity. In battery systems made from a series of connected cells, this leads to an imbalance in the state of charge (SoC) and SOH state of health during use. 
The battery state can be distinguished into two categories: those that can be directly measured, such as voltage, current, temperature, etc.; and those that cannot be directly measured[1] but can only be estimated using specific techniques, such as the battery's state of charge (SOC) and state of health (SOH).

This thesis mainly focuses on the SOC from the battery state diagram.


\section*{Structure of Thesis}
The practical application and simple hardware touch have received more emphasis in my thesis. The project is cross-disciplinary and incorporates the use of software, hardware, modeling using mathematics, and analytics. I'm keeping five objectives for my thesis, which is the project's topic because the project is much larger and more standardized.
The five modules in this report are the areas that I've illustrated.

The purpose and format of my thesis are described in the parts that follow.

\section*{Wireless Communication Environment for BMS}
The biggest problem in today's EV technology is maintaining the weight limit to save energy, in conventional EVs the weight of the wires used for sensing the matter and sense is huge.  Reducing all those wire weights could be a magical moment for EVs. One of the objectives of this project set up safe and sophisticated wireless communication BMS. I have attempted to describe the wireless architecture for the BMS and BLE modules design in chapter \ref{chap:BLE}. I have also tried to explain the general considerations for RF layout design and placements. 

\section*{Hardware Architecture for the Battery Active Balancing }
Implementing high-performing lithium-ion battery systems uses a developing approach called active charge balancing. The efficiency and power constraints of the current balancing systems are addressed by the six new energetic balancing approaches suggested in this thesis. The six approaches vary yet are similar in terms of their underlying operating principles. Among six I have proposed a Type 2 Active balancing topology for this particular project. Nonetheless, They all utilize non-isolated DC/DC converters and a MOSFET switch matrix, which is a commonality among them. They can balance any number of battery system cells at high currents depending on the power flow and switch-matrix arrangement. It is possible to balance adjacent cells at the same time. The efficiency values of DC/DC converter prototypes used in the performance comparison were estimated and measured using batch numerical simulations. I have not presented any specific results regarding passive balancing by intuitive understanding active balancing proves better efficient. Because the active balancing method dose does not dissipate extra charge floating in the battery, it will convert it to current and charge/discharge to the unbalanced battery. Chapter \ref{ch:Architecture_Active_Balancing_BMS} is the complete discussion of the active balancing topologies and the hardware setup.

\section*{Current Measurement Setup }
The current measurement process is the heart of soc estimation. The accuracy of the soc estimation of the battery depends on the current measurement duration and the current measurement synchronization. It is demanded in the BMS application to synchronize the current measurement with a less noisy environment. Chapter \ref{ch:Current_Measurement} will present the significance of the current measurement time and the synchronization of the measurements.

\section*{Setup the Saftey Lab Environment for BMS Testing}
LIPO batteries are one of the best gifts for EVs, but they are much more dangerous to handle. Because of their chemical composition and voltage tolerance. LIPO batteries are less tolerant to the batteries imbalance in the battery pack. Nevertheless, engineers need to perform the BMS testing on the LIPO batteries. This could push engineers into the corner, to resolve this issue I have presented the battery modeling the implementation with lab setup in chapter 4. The first half of chapter 4 is a discussion of the mathematics and analytics behind battery modeling. The second half is practical implementation with the lab instruments.
Battery model implementation is handled through a python script, which is going to be emulated in the instruments.

\section*{SoC Estimation Algorithms for the Battery}
To perform active balancing on the batteries we need to estimate the battery soc depending on the current flowing in/out to the batteries. Conventionally soc estimation is done through the coulomb count method (integrating the current flowing in the battery). But the coulomb count method is not so robust to eliminate noise in the system for advanced BMS we need more robust algorithms. In chapter \ref{ch:Battery_SOC_Estimation_Algorithms} I have presented different types of SOC algorithms and differentiated from each. Among several algorithms that I describe, my focus is on the Kalman algorithm and its sub-algorithms (Extended Kalman Algorithm(EKF), unscented Kalman algorithm (UKF), Adaptive unscented Kalman algorithm (AUKF)). Why there are so many algorithms and competition to estimate the battery soc? well, it is a debatable choice of soc estimation algorithms depending on the user application and the setup. Chapter \ref{ch:Battery_SOC_Estimation_Algorithms} is also immune to present the discussion about the machine learning algorithms and their future of it in EVs.