\section{Kalman Filter}\label{sec:KalamanFilter}
In essence, a Kalman filter is a system of recursive equations that implements a predictor-corrector type estimator. It creates an ideal guess of the system state based on the input control and output measurements; as a result, it is typically employed when the system state cannot be directly monitored and needs to be predicted optimally from the output measurements \cite{SOC_Estimation_KalmanFilter_Ahmad}.
In this section \ref{sec:KalamanFilter}, the extended Kalman filter (EKF) for non-linear systems is explained, followed by the discrete Kalman filter for linear systems, and lastly a state space model for the Li-Ion battery is built up. To assess the reliability of the suggested method, comparisons are also made between SOC values calculated by Ampere-counting and SOC values estimated by EKF \cite{SOC_Estimation_KalmanFilter_Liu}.

\subsection{The Discrete Kalam Filter}\label{sec:Discrete_Kalam_Filter}
This section outlines the fundamental design of a Kalman filter, which takes measurements and estimates the system state at discrete time intervals \cite{LIPO_Batt_Parameters_identification_Rahmoun}.

The linear difference stochastic equation \ref{eq:Discrete_State_Equation} governs a discrete time-controlled process, and the problem of estimating the state vector $x_k \in \mathfrak{R}^m$ of this process is handled by the discrete Kalman filter.

\begin{equation}\label{eq:Discrete_State_Equation}
    x_{k+1} = A_k x_k + B_k u_k + w_k
\end{equation}
where the measurement vector $y_k \in \mathfrak{R}^m $ is given by \ref{eq:Discrete_Measurement_Output_Equation}.
\begin{equation}\label{eq:Discrete_Measurement_Output_Equation}
    y_{k} = C_k x_k + D_k u_k + v_k
\end{equation}

Equation \ref{eq:Discrete_State_Equation} is referred to as a "state equation" or "process equation".This equation explains the dynamics, stability, controllability, and disturbance sensitivity of the system.
The control input to the system is $u_k \in \mathfrak{R}^p$, and $w_k \in \mathfrak{R}^n$ is a random variable that represents the "process noise" \cite{SOC_Estimation_KalmanFilter_Ahmad}.
The output equation of the discrete system is represented in the equation \ref{eq:Discrete_Measurement_Output_Equation}, the output state equation defines the dependency on the state vector $x_k$, control input $u_k$ and $v_k \mathfrak{R}^{m}$, Which models the measurement noise in the system.

The matrix $A_k \in \mathfrak{R}^{nxn}$ describes the system dynamics and
relates the state at the previous time step $k-1$ to the state at
the current time step k when the control input $u_k$ is zero. The
matrix $B_k \in \mathfrak{R}^{nxp}$ relates the control input $u_k$ to the state $x_k$.
The matrices $C_k \in \mathfrak{R}^{mxn}$ and $D_k \in\mathfrak{R}^{mxp}$ relate the measurement
$y_k$ to the state $x_k$ and control input $u_k$. All these matrices can be time-varying which can be extensively described through the Extended Kalman Filter.

Given a system model, a known control input $u_k$, a known measurement $y_k$, and certain assumptions, the Kalman filter approach can provide the most accurate estimation of the unmeasured state value $x_k$.
First, it is assumed that both wk and $v_k$ are mutually uncorrelated white Gaussian random processes with zero mean and known values for the covariance matrices.

\begin{equation}\label{eq:Discrete_Noise_Covariance}
    P(w) \sim N(0,Q) ; \\
    P(v) \sim N(0,R) \\
\end{equation}
Process noise covariance Q and measurement noise covariance R might change with each time step, but here we assume that they are constants  \cite{SOC_Estimation_KalmanFilter_Ahmad}.
The system must also be "observable," which means that it must be feasible to infer its state from its output. This criterion is met by the system we are developing.

\begin{algorithm}[H]\label{algo:PowerAnalyzer_Modeling}
    \DontPrintSemicolon
    \SetAlgoLined
    
    \CommentSty{Initialize}\\
    $A, B, C, D$\\
    $x_0, P_0, Q, R$ \\
    \While{1}{
        $State 1 Prediction: updateInput(u_k)$\\

         $x_{k+1/k}(x_k,u_k)$\\
         $P_{k+1/k}(P_k)$\\

        \noindent $State 2 Update: updateMeasurment(y_k)$\\

            \indent $Kalman Gain : K_k(yk,P_{k+1/k})$ \\
            $Update state :$ \\
                    $x_{k+1}(K_k,y_k,x_{k+1/k})$\\
                    $P_{k+1}(K_k,P_{k+1/k})$\\
    }
    \caption{General Discrete Kalman Filter Algorithm}
\end{algorithm}

\begin{algorithm}[H]\label{algo:PowerAnalyzer_Modeling}
    \DontPrintSemicolon
    \SetAlgoLined
    
    \CommentSty{Initialize}\\
    $Initial Estimate : x_{0/0}$\\
    $Error Covariance : P_0, Q, and R,k=0$ \\
    \While{TRUE}{
        $State 1 Prediction: updateInput(u_k)$\\

        $Determine State$
        $xk + 1/k = Adxk/k + Bduk$
        $Determine Output$
        $yk + 1 = Cdxk/k + Dduk$
        $Determine Error Covariance$
        $Pk + 1/k = AdPk/kAdT + R$

        $State 2 Correction: updateMeasurment(y_k)$\\

        $Determine Kalman Gain$
        $Kk + 1 = Pk + 1/kCdT [CdPk + 1/kCdT + Q]–1$
        $Employ Correction on Prediction of States$
        $xk + 1/k + 1 = xk + 1/k + Kk + 1 [yk + 1 – yk + 1] $
        $Determine Error Covariance$
        $Pk + 1/k + 1 = [I – Kk + 1Cd] Pk + 1/k$
    }
    \caption{General Discrete Kalman Filter Algorithm}
\end{algorithm}
    \begin{figure}
        \centering
        \begin{tikzpicture}[node distance=2cm]
            \node (start) [startstop] {
                \makecell[l]{Initial Estimate $x_{0/0}$ and \\ 
                            Error Covariance $P_0$, Q, and R}
            };
            \node (in1) [io, below of=start,yshift=-1cm] {
                \makecell {
                    \textbf{State Prediction} \\
                    Determine State\\
                    $\hat{x_{k + 1/k}} = A \hat{x_{k/k}}  + B u_k$\\
                    Determine output\\
                    $\hat{y_{k + 1}} = C \hat{x_{k/k}} + D u_k$\\
                }
            };
            \node (pro1) [process1, below of=in1,yshift=-2.25cm] {
                \makecell{ 
                    \textbf{Error Covariance} \\
                    $P_{k + 1/k} = A P_{k/k} A^T + R$\\
                }
            };
            \node (pro2) [process2, right of=pro1, xshift=5cm] {
                \makecell{ 
                    \textbf{ Correction} \\
                   Determine Kalman Gain\\
                   $K_{k + 1} = P_{k + 1/k} C^T [C P{k + 1/k} C^T + Q]^{–1}$\\
                   Employ Correction on Prediction of States\\
                   $\hat{x_{k + 1/k + 1}} = \hat{x_{k + 1/k}} + K_{k + 1} [y_{k + 1} –  \hat{y_{k + 1}}]$ \\
                   Determine Error Covariance\\
                   $P_{k + 1/k + 1} = [I – K_{k + 1} C] P{k + 1/k}$\\
                }
            };
            \draw [arrow] (start) -- (in1);
            \draw [arrow] (in1) -- (pro1);
            \draw [arrow] (pro2) |- (in1);
            \draw [arrow] (pro1) -- (pro2);
        \end{tikzpicture}
        \caption{Discrete Kalman Filter Algorithm Flow Chart}
        \label{algo:Discrete_Kalman_Filter_Algorithm_Flow_Chart}
\end{figure}


\begin{itemize}
    \item \textbf{Initialization :}  A, B, C, and D represent the system and have to be defined. Then the state vector $x_0$ and its associated covariance vector $P_0$ for $K=0$ are initialized. Both perturbations, w and v are uncorrelated white Gaussian random processes with known value covariance matrices Q and R are initialized at this step of the algorithm.
    \item \textbf{Prediction :} For each iteration k newly acquired input data, u is injected into the system, Based on the calculated prior state $x_k and x_{k+1/k}$, the state vector is predicted along with predicted covariance $P_{k+1/k}$.
\end{itemize}

\begin{equation}\label{eq:Discrete_Prior_State_Equation}
    x_{k+1/k} = A_k x_k + B_k u_k 
\end{equation}
\begin{equation}\label{eq:Discrete_Prior_Covariance_Equation}
    P_{k+1/k} = A_k P_k A_k^{T} + Q
\end{equation}

If the system is stable then $A_k P_k A_k^{T}$ is diminished and reduces the uncertainty of the state estimation over time. The process noise term Q always increases the
uncertainty because wk cannot be measured. The second estimated $x_{k+1}$ tunes up the first estimated $x_{k+1/k}$ (prior state) after measuring the system output $y_k$. The state and error covariance $x_{k+1}$ and $P_{k+1}$
are more accurate than $x_{k+1/k}$ and $P_{k+1/k}$ as they involve the information from the measurement $y_k$.
\begin{itemize}
    \item \textbf{Corrction and Update :} A correction factor () equal to the system output is added to the new measurement to provide fresh information.
\end{itemize}

\subsection{Extended Kalman Filter }

The extended Kalman filter is the Kalman filter for nonlinear systems. With the extended Kalman filter technique, a linearization procedure is carried out at each time step to approximate the nonlinear system with a linear time-varying system. The extended Kalman filter for the real nonlinear system is produced by using the linear time-changing system in a Kalman filter. Similar to a Kalman filter, an extended Kalman filter assumes that the process noise and sensor noise are independent, zero-mean Gaussian noises and uses the measured input and output to get the minimum mean squared error estimate of the true state %\cite{ADIJ_MARTIN2017}.

In the battery pack system Equation 28 and 29, the system state variables are defined as $x_1(t) = SOC_0$ and $x_2(t) = V_{cs}$. The input is defined as u(t) = $i_{batt}$ and the output is y(t) = $V_batt$. The battery pack system Equation 28 and 29 can be rewritten as :
\begin{equation}
    \dot{x}  = f(x,u) + w
\end{equation}
\begin{equation}
    y  = g(x,u) + v
\end{equation}
where $x = [x_1,x_2]^T$, The functions $f(x,u)  and g(x,u) are :$
\begin{equation}\label{eq:Batt_Kalman_State_function}
    f(x,u) =  \begin{bmatrix}
                    \frac{u}{k C_{OTC}} \\
                    -\frac{1}{R_{OTC} C_{OTC}} x_2 + \frac{1}{C_{OTc}} u
               \end{bmatrix}  
\end{equation}
\begin{equation}\label{eq:Batt_Kalaman_output_function}
    g(x,u) = k x_1 + x_2  + R_0 u + d
\end{equation}
If the functions f(x,u) and g(x,u) are linearized by a first-order, Taylor 
series expansion, at each sample step about the current operating point, 
the linearized model is 
\begin{equation}\label{eq:Batt_Kalman_State_function_tylor_expansion}
    \delta \dot{x} = A_k \delta x + B_k \delta u
\end{equation}
\begin{equation}\label{eq:Batt_Kalaman_output_function_tylor_expansion}
    \delta y = C_k \delta x + D_k \delta u
\end{equation}

where ;
\begin{equation}\label{eq:Batt_Kalman_function_tylor_expansion_Ak}
    A_k = \frac{d f(x,u)}{d x} =  \begin{bmatrix}
                                        0 & 0 \\
                                        0 & -\frac{1}{R_{OTC} C_{OTC}} 
                                  \end{bmatrix}                          
\end{equation}
\begin{equation}\label{eq:Batt_Kalman_function_tylor_expansion_Bk}
    B_k = \frac{d f(x,u)}{d x} =  \begin{bmatrix}
                                        -\frac{1}{K C_{cb}} \\
                                         -\frac{1}{ C_{OTC}} 
                                  \end{bmatrix}                          
\end{equation}
\begin{equation}\label{eq:Batt_Kalman_function_tylor_expansion_Ck}
    C_k = \frac{d f(x,u)}{d x} =  \begin{bmatrix}
                                    K & 1\\  
                                  \end{bmatrix}                          
\end{equation}
\begin{equation}\label{eq:Batt_Kalman_function_tylor_expansion_Dk}
    D_k = \frac{d f(x,u)}{d x} =  [R_0]                         
\end{equation}

The battery model represented by equation \ref{eq:Batt_Kalman_State_function_tylor_expansion} and \ref{eq:Batt_Kalman_function_tylor_expansion_Bk} can be discretized as
\begin{equation}\label{eq:Batt_Kalman_State_Prediction}
    x_{k+1} = A_d x_k + B_d u_k 
\end{equation}
\begin{equation}\label{eq:Batt_Kalman_Output_Prediction}
    y_{k+1} = C_d x_k + D_d u_k
\end{equation}

where $A_d \simeq  E + T_c A_k, B_d \simeq  T_c B_k$, E is the unit matrix and $T_c$ is the sampling 
period, and $C_d \simeq  C_k, D_d \simeq  D_k$ . 
\begin{figure}
    \centering
    \includegraphics[width=0.5\textwidth]{Chap07/Figures/Kalman_principle.PNG}
    \caption{Kalman Filter Principle}
    \label{fig:Kalman_Filter_Principle}
\end{figure}